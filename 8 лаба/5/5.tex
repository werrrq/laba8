\documentclass[pdf,hyperref={unicode}, aspectratio=43, serif,11pt]{beamer}
\usepackage[T2A]{fontenc}
\usepackage[english, russian]{babel}  
\usepackage{graphicx}
            
%Красная строка в первом абзаце
\usepackage{indentfirst}
%Величина отступа красной строки
\setlength{\parindent}{12.5 mm}

%Межстрочный интервал
%\def\baselinestretch{1.5}
\usepackage{setspace}
\setstretch{1}

\title[физмат]{Физико-математический факультет}
\author{Козин А.В.}
\date{19 июня 2024 года}
\institute[]{Орловский государственный
университет имени И.\,С.~Тургенева}
\def\baselinestretch{1}

\usefonttheme[onlymath]{serif}
\usepackage{beamerthemesplit}

%тема оформления
\usetheme{Madrid}%Warsaw

%цветовая гамма
\usecolortheme{seahorse}%whale
\begin{document}
\begin{frame}
\titlepage{}
\end{frame}
\begin{frame}{Первый декан физмата}
    \begin{center}
        \includegraphics{0.jpg}
    \end{center}
\end{frame}
\begin{frame}{Июнь 1935 года}
    \begin{figure}[h]
        \begin{tabular}{ll}
            \includegraphics[scale=0.3]{1.jpg}
            &
            \includegraphics[scale=0.3]{2.jpg}
        \end{tabular}
        \caption{Первый выпуск и Диплом №1}
        \label{Fig:Race}
    \end{figure}
\end{frame}
\begin{frame}{Герои Великой Отечественной войны}
    \begin{figure}[h]
        \begin{tabular}{ll}
            \includegraphics[scale=1]{3.jpg}
            &
            \includegraphics[scale=1]{4.jpg}
        \end{tabular}
        \label{Fig:Race}
    \end{figure}
\end{frame}
\begin{frame}{Сергей Митрофанович Клименко (1941 г.-1947 г.)}
    \begin{center}
        \includegraphics{5.jpg}
    \end{center}
\end{frame}
\begin{frame}{Научная работа Павла Степановича Кудрявцева}
    \begin{figure}[h]
        \begin{tabular}{ll}
            \includegraphics[scale=0.8]{6.jpg}
            &
            \includegraphics[scale=0.8]{7.jpg}
        \end{tabular} \\
        Написал трехтомную «Историю всемирной физики», а в 1960-1970 гг. в серии «Жизнь замечательных людей» издал свои книги о физиках И. Ньютоне, Д. Максвелле, Э. Торричелли.
        \label{Fig:Race}
    \end{figure}
\end{frame}
\begin{frame}{Деканы-герои}
    \begin{figure}[h]
        \begin{tabular}{ll}
            \includegraphics[scale=0.8]{8.jpg}
            &
            \includegraphics[scale=0.8]{9.jpg}
        \end{tabular}
        \label{Fig:Race}
        \caption{С.М. Горшенин (1948 г.-1953 г.) и А.П. Алексеев (1953 г.-1957 г.)}
    \end{figure}
\end{frame}
\begin{frame}{30 июля 1957 год - новое здание педагогического института}
    \begin{center}
        \includegraphics[scale=0.5]{10.jpg}
    \end{center}
\end{frame}
\begin{frame}{Достижения}
    \begin{figure}[h]
        \begin{tabular}{ll}
            \includegraphics[scale=0.5]{11.jpg}
            &
            \includegraphics[scale=0.5]{12.jpg}
        \end{tabular}
        \label{Fig:Race}
    \end{figure} \\
    На ежегодных выставках творческих работ студентов физматовцы по праву занимали призовые места. Приборы, изготовленные ими, обычно передавались в подшефные школы.
\end{frame}
\begin{frame}
    \begin{figure}[h]
        \begin{tabular}{ll}
            \includegraphics[scale=0.5]{13.jpg}
            &
            \includegraphics[scale=0.5]{14.jpg}
        \end{tabular}
        \label{Fig:Race}
    \end{figure} \\
    физматовцы-альпинисты на перевале имени Т.Н. Грановского: Филонов А. (слева), Шерман В. (справа), Хрипунов В. (в центре)
\end{frame}
\begin{frame}{1957 — 1964}
    \begin{center}
        \includegraphics{15.jpg} \\
        \caption{Александров Василий Александрович - декан физмата}
    \end{center}
\end{frame}
\begin{frame}{Геннадий Андреевич Зюганов - выпускник физмата}
    \begin{figure}[h]
        \begin{tabular}{ll}
            \includegraphics[scale=0.6]{16.jpg}
            &
            \includegraphics[scale=0.5]{17.jpg}
        \end{tabular}
        \label{Fig:Race}
    \end{figure}
\end{frame}
\begin{frame}{Движение студенческих строительных отрядов (ССО)(1-я половина 1970-х гг.)}
    \begin{center}
        \includegraphics[scale=0.6]{18.jpg}
    \end{center}
\end{frame}
\begin{frame}{Декан факультета во второй половине 1970-х годов}
    \begin{center}
        \includegraphics[scale=1]{19.jpg} \\
        \caption{Дмитрий Григорьевич Курбан}
    \end{center}
\end{frame}
\begin{frame}{1969 г.-1974 г.}
    \begin{center}
        \includegraphics[scale=1]{20.jpg} \\
        \caption{Вениамин Константинович Инножарский(декан физмата)-одним из первых в нашем вузе стал преподавать программирование}
    \end{center}
\end{frame}
\begin{frame}{1983 г.-2003 г.}
    \begin{center}
        \includegraphics[scale=1]{21.jpg} \\
        \caption{Байдак Геннадий Васильевич(декан физмата)}
    \end{center}
\end{frame}
\begin{frame}{События}
    \begin{center}
        \includegraphics[scale=0.3]{22.jpg}
    \end{center}
\end{frame}
\begin{frame}{Руководитель факультета в настоящее время}
    \begin{center}
        \includegraphics[scale=1]{23.jpg} \\
        \caption{Т.Н. Можарова(профессор, кандидат физико-математических наук)}
    \end{center}
\end{frame}
\begin{frame}{Таблица всех деканов физмата}
    \begin{tabular}{|c|S|}
        \hline
        Поподько Т.И. & 1939 1944 \\
        Еремин П.Д. & 1939-1940 гг.\\
        Клименко С.М. & 1941-1947 гг. \\
        Горшенин С.М. & 1948-1953 гг. \\
        Алексеев А.П. & 1953-1957 гг. \\
        Александров В.А. & 1957 -1964 гг. \\ 
        Курбан Д.Г. & вторая половина 1970-х \\
        Инножарский В.К. & 1969-1974 гг. \\
        Байдак Г.В. & 1983-2003 гг. \\
        Можарова Т.Н. & наше время \\
        \hline
    \end{tabular}
\end{frame}

\end{document}