\documentclass[12pt]{article}
\usepackage[utf8]{inputenc}
\usepackage[english,russian]{babel}
\usepackage{amssymb, amsmath, ragged2e, fancyhdr}
\linespread{0.8}

\pagestyle{empty}

\begin{document}

\setcounter{equation}{2}
\pagestyle{fancy}
\fancyhf{}
\lhead{§2j}
\chead{МЕТОД РАЗДЕЛЕНИЯ ПРЯМЫХ}
\rhead{201}
\renewcommand{\headrulewidth}{0pt}

\raggedright
и граничными условиями

\begin{equation}
\left\
    \begin{array}{l}
u(o,t)=\mu_1(t),\\
u(o,t)=\mu_1(t)
\end{array}
\right\} (t\geq0).
\end{equation}

\hspace{0.5cm}
Изучение общей первой краевой задачи начнем с решения следующей простейшей задачи I:

\hspace{0.5cm}
\textit{найти непрерывное в замкнутой области (0\leq x\leq l,0\leq}\\

\textit{\leq t\leq T) решение однородного уравнения}


\begin{center}
$u_t=a^2u_{xx},\hspace{0.5cm}0<x<l,\hspace{0.5cm}o<t\leq T,$
\end{center}

удовлетворяющее начальному условию
\begin{center}
$u(x,0)=\varphi(x),\hspace{0.5cm}0\leq x\leq l$
\end{center}

и однородным граничным условиям 

\begin{center}
$u(0,t)=0,\hspace{0.5cm}u(l,t)=0,\hspace{0.5cm}0\leq t\leq T. $
\end{center}

\hspace{0.5cm}
\textnormal{Для решения этой задачи рассмотрим, как принято в мето-\\де разделения переменных, сначала основную вспомогательную\\ задачу:}

\hspace{0.5cm}
\textit{найти решение уравнения}

\begin{center}
$u_t=a^2u_{xx},$
\end{center}

\textit{не равное тождественно нулю, удовлетворяющее однородным \\ граничным условиям}

\begin{center}
$u(o,t)=0,\hspace{0.5cm}u(l,t)=0$
\end{center}

\textit{и представимое в виде}

\begin{center}
$u(x,t)=X(x)T(t),$
\end{center}

\textit{где X(x)-функция только переменного x, T(t)-функция \\ только перемнного t.}

\hspace{0.5cm}
\textnormal{Подставляя предпологаемую форму решения (6) в уравне-\\ние (4) и производя деление обеих частей равенства на $a^2XT$,\\получим:}

\begin{center}
$\frac{1}{a^2}\frac{T^`}{T}=\frac{X^"}{X}=-\lambda,$
\end{center}

\textnormal{где $\lambda= const$, так как левая часть равенства зависит только \\от t, а правая - только от x.}

\hspace{0.5cm}
\textnormal{Отсюда следует, что}

\begin{center}
$X"+\lambda X= 0,\\T'+a^2\lambda T = 0.$
\end{center}

\textnormal{Граничные условия (5) дают:}

\begin{center}
$X(0)=0,\hspace{0.5cm}X(l)=0.$
\end{center}

\newpage
\lhead{202}
\chead{УРАВНЕНИЯ ПАРАБОЛИЧЕСКОГО ТИПА}
\rhead{[ГЛ III}
\renewcommand{\headrulewidth}{0pt}

\textnormal{Таким образом, для определения функции X(x) мы получили \\ задачу о собственных значениях (задачу Штурма - Лиувилля)}

\begin{center}
$X"+\lambda X=0,\hspace{0.5cm}X(0)=0,\hspace{0.5cm}X(l)=0,$
\end{center}

\textnormal{исследованную при решении уравнения колебаний в главе II\\ (см.\textsection3,п.1).При этом было показано, что только для зна-\\чений параметра $\lambda$, равных }

\begin{center}
$\lambda_n=(\frac{\pi n}{l})^2\hspace{0.5cm}(n=1,2,3,...),$
\end{center}

\textnormal{существуют нетривиальные решения уравнения (8), равные }

\begin{center}
$X_n(x)=sin\frac{\pi n}{l}x.$
\end{center}

\textnormal{Этим значениям $\lambda_n$ соответствуют решения уравнения (8`)}

\begin{center}
$T_n(t)=C_ne^{-a^2\lambda_n t},$
\end{center}

\textnormal{где $C_n$- не определенные пока коэффициенты.\\ \hspace{0.5cm} Возвращаясь к основной вспомогательной задаче, видим, что \\ функции}

\begin{center}
$u_n(x,t)=X_n(x)T_n(t)=C_ne^{-a^2\lambda_n t}sin\frac{\pi n}{l}x,$
\end{center}

\textnormal{являются частными решениями уравнения (4), удовлетворяю-\\щими нулевым граничным условиям.\\ \hspace{0.5cm}Обратимся теперь к решению задачи (I). Составим фор-\\мально ряд}

\begin{center}
$u(x,t)=\displaystyle\sum_{n=1}^{\infty}C_ne^{-(\frac{\pi n}{l})^2a^2t}sin\frac{\pi n}{l}x.$
\end{center}

\textnormal{Функция $u(x,t)$ удовлетворяет граничным условиям, так как \\ им удовлетворяют все члены ряда. Требуя выполнения началь-\\ных условий, получаем:}

\begin{center}
$\phi(x)=u(x,0)=\displaystyle\sum_{n=1}^{\infty}C_nsin\frac{\pi n}{l}x,$   
\end{center}

\textnormal{т.е. $C_n$ являются коэффициентами Фурье функции $\phi(x)$ при \\ разложении ее в ряд по синусам на интервале (0,l):}

\begin{center}
$C_n=\phi_n=\frac{2}{l}\int\limits_0^l\phi(\xi)sin\frac{\pi n}{l}\xi\bullet d\xi$   
\end{center}

\textnormal{\hspace{0.5cm}Рассмотрим теперь ряд (15) с коэффициентами $C_n$, опреде-\\ ляемыми по формуле (17), и покажем, что этот ряд удовлетво-\\ряет всем условиям задачи (I). Для этого надо доказать, что}
\end{document}
