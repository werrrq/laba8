\documentclass[13pt]{article}
\usepackage[english, russian]{babel}
\usepackage{epsf,amsbsy,amscd,amsmath,amsfonts,euscript,amssymb,latexsym, textcase, geometry, ifthen, soul,graphicx,float}
\usepackage{caption}
\usepackage{subcaption}
\geometry{top=2cm} 
\geometry{bottom=2cm} 
\geometry{left=2cm} 
\geometry{right=2cm}
\parindent=1.27cm 

\title{ПОСТРОЕНИЕ ГРАФА ПЕРЕХОДОВ ПОСЛЕДОВАТЕЛЬНОСТНОЙ СХЕМЫ\\ С ПРИМЕНЕНИЕМ SAT-РЕШАТЕЛЯ}
\author{\LARGEИванов Иван Иванович \\
\LARGEкандидат технических наук, доцент \\
\LARGEдоцент, Томский государственный университет \\
\LARGEРоссия, Томск \\ Петров Петр Петрович \\
\LARGEстудент, Томский государственный университет \\
\LARGEРоссия, Томск}
\begin{document}
\maketitle
\begin{Large}
Аннотация. Рассматривается подход к построению графа
переходов последовательностной схемы. Исследуется метод,
основанный на использовании SAT-решателя. Для построения графа
переходов определяются возможные переходы между состояниями
последовательной схемы и используются предварительные
вычисления, основанные на троичном и двоичном моделировании,
значительно сокращающие объем вычислений. Также
рассматривается построение на основе графа переходов
последовательностей, обеспечивающих заданные переходы схемы.
Компьютерные эксперименты показали эффективность
предложенного метода построения графа переходов с применением
SAT-решателя.
\parКлючевые слова: граф переходов, троичное моделирование,
SAT-решатель, последовательностная схема, переходная
последовательность.\\
\parВ работе рассматривается подход к построению графа переходов
синхронной последовательностной схемы. Построение графа,
основанное на использовании SAT-решателя, исследуется подробно.
Представлены результаты компьютерных экспериментов для
предложенного метода построения графа переходов, применяющего
SAT-решатель. Также кратко рассматривается решение задачи
построения последовательности входных векторов, обеспечивающей
переход схемы в одно из состояний заданного множества, по графу
переходов. 
\newpage
\parРассмотрим синхронную последовательностную схему с $n$ 
входами, $m$ выходами и $p$ элементами памяти (триггерами).
$X = \{{x_1, ..., x_n}\} $ - входные переменные схемы, $Y = \{{y_1, ..., y_m}\}$ - ее
выходные переменные, $Z = \{{z_1, ..., z_p}\} $ – внутренние переменные
схемы.
\parНазовем графом переходов последовательностной схемы
ориентированный граф, у которого вершины сопоставлены
состояниям схемы и есть дуга из вершины $i$ в вершину $j$ тогда и
только тогда, когда в схеме существует одношаговый переход из
состояния, соответствующего вершине $i$, в состояние,
соответствующее вершине $j$, при каких-либо значениях входных
переменных.
\parНа рисунке 1 представлена комбинационная составляющая С
последовательностной схемы. При построении графа переходов
рассматривается только та часть схемы, которая необходима для
получения функций переходов, выходы схемы не рассматриваются.
Поэтому структурное описание комбинационной составляющей,
используемой для получения графа переходов, упрощается (рисунок
2).\\\end{Large}
\begin{figure}[h!]
\centering
\begin{subfigure}{.5\textwidth}
  \centering
  \includegraphics[height=6cm]{Снимок.PNG}
  \caption{Рисунок 1 – Ирис setosa}
  \label{рисунок 1}
\end{subfigure}%
\begin{subfigure}{.5\textwidth}
  \centering
  \includegraphics[height=6cm]{Снимок1.PNG}
  \caption{ Рисунок 2 - Ирис virginica}
  \label{fig:sub2}
\end{subfigure}
\label{рисунок 2}
\end{figure}
\begin{Large}
\parВ схеме с рисунка 2 можно исключить все элементы, не
связанные с ее выходами, то есть с входами триггеров
последовательностной схемы.

Система функций переходов последовательностной схемы имеет
вид:
\newpage
$$z^t_j=\psi_j(x^t_1,x^t_n,z^{t-1}_1,...,z^{t-1}_p),j=1,p.$$
\parБудем представлять полное состояние схемы вектором $(\alpha, \delta)$, где $\alpha$ - вектор значений входных переменных X, а $ \delta$ - вектор значений внутренних переменных Z.
\parДвоичный вектор $\tau^i=(\tau^i_1,...,\tau^i_p)$ значений переменных Z будем называть кодом состояния $q_1$.$Q={q_1,...,q_t}$,где $t=2^p$, - множество всех состояний схемы.
\parРассмотрим общий подход к построению графа переходов последовательностной схемы предложенный в работах [1, 2] и других.
\parПредставленные свойства определены в [3].
\parРассмотрим подробнее следующее свойство, сформулированное в [3], используемое на 2-ом шаге сокращения вычислений.
\\Пусть выполнено точное троичное моделирование функций переходов системы (1) на векторе $(\alpha, \delta)$ представляющем полное состояние, и получен вектор значений внутренних переменных $ \delta'$.$ \delta'$ представляет минимальный покрывающий интервал множества булевых векторов значений переменных Z, а не точное множество этих векторов. Таким образом, множество состояний схемы достижимых за один шаг из множества $N(\alpha,\delta)$ может быть подмножеством множества $N(\delta')$.
\par{\ Результаты экспериментов}
\parДля проверки эффективности предложенного метода построения графа переходов последовательностной схемы, использующего SAT-решатель, были проведены эксперименты на бенчмарках. Эксперименты проводились на контрольных примерах (бенчмарках) ISCAS’89, представляющих последовательностные схемы. Для оценивания работы исследуемого метода при проведении экспериментов для различных бенчмарок измерялось время построения графа переходов, а также процент определяемых значений элементов матрицы M на каждом шаге предварительных вычислений. Результаты компьютерных экспериментов представлены в таблице 1. 
\end{Large}
\begin{table}[h!]
    \centering
    \caption{Результаты построения графов переходов}
    \begin{tabular}{|p{1.1cm}|p{1cm}|p{0.9cm}|p{1.2cm}|p{1.2cm}|p{1.3cm}|p{1.7cm}|p{1.7cm}|p{1.6cm}|} 
    \hline
    Бенч-&Входы&Вы&Элемен&Элемен&Среднее&Определен&Определен&Соотноше\\
    марки&(\#)&ходы&ты па-&ты&время&ные&ные&ние\\ 
    &&(\#)&мяти&(\#)&построе&переходы&0 и 1&в матри\\
    &&&(\#)&&ния&на щаге 1&на шаге 2&це М\\
    &&&&&графа&(только 1)&(только 0)&(#0; #1)\\
    &&&&&(сек.)&(\%)&(\%)&\\ \hline
    S27&4&1&3&10&0,01&17,19&31,25&31;33\\ \hline
    S386&7&7&6&159&1,01&2,15&92,77&3800;296\\ \hline
    S832&18&19&5&287&7,60&11,13&61,23&710;314\\ \hline
    S510&19&7&6&211&21,84&2,46&997,36&3995;101\\ \hline
    S1488&8&19&6&653&4,82&3,13&82,25&3369;727\\ \hline
    \end{tabular}
    \label{таблица 1}
\end{table} 
\begin{Large}
\parВ таблице представлены следующие данные: имя бенчмарки, 
количество входов, выходов, элементов памяти и элементов схемы; 
среднее время построения графа переходов схемы (по трем 
экспериментам); процент существующих одношаговых переходов в 
графе, определенных на шаге 1 предварительных вычислений, 
процент не существующих одношаговых переходов, определенных на 
шаге 2 предварительных вычислений, и соотношение 0 и 1 в 
полученной матрице M.
\parШаги предварительных вычислений в экспериментах для
рассмотренных бенчмарок определили от 48,44% до 99,82% 
одношаговых переходов (существующих и не существующих в 
графе). Шаг 2 предварительных вычислений, выполненный с 
помощью троичного моделирования, позволил определить большую 
часть одношаговых переходов, отсутствующих в графе. Значительная 
часть всех одношаговых переходов графа вычислена с помощью 
шагов 1 и 2 предварительных вычислений. Графы переходов для схем 
среднего размера построены за несколько секунд.
\begin{center}
  \textbf{Список литературы}
\end{center} 
\\
\\1. Иванов И.И. Построение графа переходов последовательностной 
схемы // Современные проблемы физико-математических наук: 
материалы IV Всероссийской науч.-практ. конф. с международным 
участием, (22 – 25 ноября 2018 г., г. Орёл). – Орел: ОГУ им. И.С. 
Тургенева, 2018. – Ч. 1. – С. 230 – 235. 
\\2. Ivanov I. Three-Value Simulation of Combinational and Sequential
Circuits and its Applications // 2020 IEEE Moscow Workshop on 
Electronic and Networking Technologies (MWENT). Moscow, Russia. 
11−13 March 2020. – 7 pp.
\\3. Иванов И.И. Интервальные расширения булевых функций и 
троичное моделирование последовательностных схем // Таврический 
научный обозреватель. – 2017. – №5 (22). – С. 208220.
\\4. Иванов И.И. Троичное моделирование комбинационных и 
синхронных последовательностных схем // Современные проблемы 
\newpage

физико-математических наук / материалы V Всероссийской науч.-
практ. конф. с международным участием, (26 – 29 сентября 2019 г., г. 
Орёл). – Орел: ОГУ им. И.С. Тургенева, 2019. – С. 274 – 281.
\end{Large}
\end{document}